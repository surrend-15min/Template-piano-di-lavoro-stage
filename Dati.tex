%----------------------------------------------------------------------------------------
%   USEFUL COMMANDS
%----------------------------------------------------------------------------------------

\newcommand{\dipartimento}{Dipartimento di Matematica ``Tullio Levi-Civita''}

%----------------------------------------------------------------------------------------
% 	USER DATA
%----------------------------------------------------------------------------------------

% Data di approvazione del piano da parte del tutor interno; nel formato GG Mese AAAA
% compilare inserendo al posto di GG 2 cifre per il giorno, e al posto di 
% AAAA 4 cifre per l'anno
\newcommand{\dataApprovazione}{Data}

% Dati dello Studente
\newcommand{\nomeStudente}{Samuele}
\newcommand{\cognomeStudente}{Vignotto}
\newcommand{\matricolaStudente}{1161712}
\newcommand{\emailStudente}{samuele.vignotto@studenti.unipd.it}
\newcommand{\telStudente}{+ 39 346 75 38 774}

% Dati del Tutor Aziendale
\newcommand{\nomeTutorAziendale}{Alessio}
\newcommand{\cognomeTutorAziendale}{Voltarel}
\newcommand{\emailTutorAziendale}{alessio.voltarel@eurosystem.it}
\newcommand{\telTutorAziendale}{+ 39 042 26 28 711}
\newcommand{\ruoloTutorAziendale}{SW Engineer \& Project Manager}

% Dati dell'Azienda
\newcommand{\ragioneSocAzienda}{Eurosystem SpA}
\newcommand{\indirizzoAzienda}{Via Isaac Newton, 21, 31020, Villorba (TV)}
\newcommand{\sitoAzienda}{https://eurosystem.it/}
\newcommand{\emailAzienda}{marketing@eurosystem.it}
\newcommand{\partitaIVAAzienda}{P.IVA 02243020266}

% Dati del Tutor Interno (Docente)
\newcommand{\titoloTutorInterno}{Prof.}
\newcommand{\nomeTutorInterno}{Tullio}
\newcommand{\cognomeTutorInterno}{Vardanega}

\newcommand{\prospettoSettimanale}{
     % Personalizzare indicando in lista, i vari task settimana per settimana
     % sostituire a XX il totale ore della settimana
    \begin{itemize}
        \item \textbf{Prima e Seconda Settimana (68 ore)}
        \begin{itemize}
            \item On-boarding;
            \item Presentazioni;
            \item Introduzione ai concetti di gestionale;
            \item Introduzione ai concetti di db;
	 \item Introduzione ai concetti di machine learning;
	 \item Introduzione ai concetti di grafi;
	 \item Introduzione ai concetti di process mining;
        \end{itemize}
        \item \textbf{Terza, Quarta e Quinta Settimana - Studio interfacciamento con Celonis (102 ore)} 
        \begin{itemize}
	 \item Studio della documentazione ufficiale e delle risorse online;
            \item Studio delle caratteristiche chiave come analisi dei processi, visualizzazione dei dati, e automazione;
	 \item Identificazione dei principali indicatori di prestazione e metriche utilizzate in Celonis;
	 \item Studio dell'architettura della piattaforma Celonis;
	 \item Analisi dei componenti principali;
	 \item Studio del processo di estrazione, trasformazione e caricamento dei dati nella piattaforma Celonis;
	 \item Creazione e gestione dei modelli di dati;
	 \item Creazione e configurazione di dashboard e report;
	 \item Applicazioni di machine learning e intelligenza artificiale in Celonis per migliorare l'analisi dei processi;
	 \item Raccolta di best practices per un utilizzo efficace della piattaforma;
        \end{itemize}
        \item \textbf{Sesta, Settima, Ottava e Nona Settimana - Implementazione, Analisi e Ottimizzazione del Modello di Dati in Celonis (136 ore)} 
        \begin{itemize}
	 \item Caricamento Dati e Affinamento
	 \begin{itemize}
		\item Estrarre i dati dai sistemi ERP e altre fonti;
		\item Pulire e trasformare i dati per adattarli ai requisiti di Celonis;
		\item Creare il modello di dati in Celonis;
		\item Affinare il modello di dati per ottimizzare le analisi;
		\item Caricare i dati nel modello di dati di Celonis;
	 \end{itemize}
	 \item Analisi dei Risultati, Conclusioni e Sviluppi Futuri
	 \begin{itemize}
		\item Progettare i prototipi per la valutazione dei dati;
	 	\item Generare dati simulati per l'istruzione dei modelli prototipo;
	 	\item Istruire i modelli prototipo con i dati estratti o simulati;
	 	\item Eseguire le analisi sui modelli prototipo;
	 	\item Valutare i risultati delle analisi;
	 	\item Correggere le criticità emerse e ripetere le analisi;
	 	\item Documentare i risultati ottenuti e preparare una relazione;
	 	\item Proporre possibili sviluppi futuri;
	 \end{itemize}
        \end{itemize}
    \end{itemize}
}

% Indicare il totale complessivo (deve essere compreso tra le 300 e le 320 ore)
\newcommand{\totaleOre}{306}

\newcommand{\obiettiviObbligatori}{
	 \item \underline{\textit{O01}}: Acquisire conoscenze di base su sistemi gestionali, database, machine learning, grafi e process mining;
	 \item \underline{\textit{O02}}: Approfondimento documentale delle piattaforme di Process Mining e Data Mining;
	 \item \underline{\textit{O03}}: Comprendere l'interfacciamento con la piattaforma Celonis e altre eventuali piattaforme;
	 \item \underline{\textit{O04}}: Caricamento dei dati estratti dall'ERP o generati ad-hoc;
	 \item \underline{\textit{O05}}: Affinamento dei dati per adattarli ai requisiti delle piattaforme;
	 \item \underline{\textit{O06}}: Analisi dei dati processati dalle piattaforme;
	 \item \underline{\textit{O07}}: Interpretazione dei risultati con l'aiuto del tutor;
	 \item \underline{\textit{O08}}: Redigere una relazione finale con spiegazioni, osservazioni, pregi e difetti delle piattaforme utilizzate;
	 \item \underline{\textit{O09}}: Proporre possibili sviluppi futuri;
	 
}

\newcommand{\obiettiviDesiderabili}{
	 \item \underline{\textit{D01}}: Identificare punti di inefficienza e possibili miglioramenti;
	 \item \underline{\textit{D02}}: Misurare le performance delle piattaforme in termini di velocità di elaborazione, precisione dei risultati e usabilità;
}

\newcommand{\obiettiviFacoltativi}{
	 \item \underline{\textit{F01}}: Studio e utilizzo di piattaforme di Process Mining diverse da Celonis;
	 \item \underline{\textit{F02}}: Approfondire conoscenze avanzate di data analysis e machine learning;
}