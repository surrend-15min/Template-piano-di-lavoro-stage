%----------------------------------------------------------------------------------------
%   USEFUL COMMANDS
%----------------------------------------------------------------------------------------

\newcommand{\dipartimento}{Dipartimento di Matematica ``Tullio Levi-Civita''}

%----------------------------------------------------------------------------------------
% 	USER DATA
%----------------------------------------------------------------------------------------

% Data di approvazione del piano da parte del tutor interno; nel formato GG Mese AAAA
% compilare inserendo al posto di GG 2 cifre per il giorno, e al posto di
% AAAA 4 cifre per l'anno
\newcommand{\dataApprovazione}{Data}

% Dati dello Studente
\newcommand{\nomeStudente}{Samuele}
\newcommand{\cognomeStudente}{Vignotto}
\newcommand{\matricolaStudente}{1161712}
\newcommand{\emailStudente}{samuele.vignotto@studenti.unipd.it}
\newcommand{\telStudente}{+ 39 346 75 38 774}

% Dati del Tutor Aziendale
\newcommand{\nomeTutorAziendale}{Alessio}
\newcommand{\cognomeTutorAziendale}{Voltarel}
\newcommand{\emailTutorAziendale}{alessio.voltarel@eurosystem.it}
\newcommand{\telTutorAziendale}{+ 39 042 26 28 711}
\newcommand{\ruoloTutorAziendale}{SW Engineer \& Project Manager}

% Dati dell'Azienda
\newcommand{\ragioneSocAzienda}{Eurosystem SpA}
\newcommand{\indirizzoAzienda}{Via Isaac Newton, 21, 31020, Villorba (TV)}
\newcommand{\sitoAzienda}{https://eurosystem.it/}
\newcommand{\emailAzienda}{info@eurosystem.it}
\newcommand{\partitaIVAAzienda}{P.IVA 02243020266}

% Dati del Tutor Interno (Docente)
\newcommand{\titoloTutorInterno}{Prof.}
\newcommand{\nomeTutorInterno}{Tullio}
\newcommand{\cognomeTutorInterno}{Vardanega}

\newcommand{\prospettoSettimanale}{
% Personalizzare indicando in lista, i vari task settimana per settimana
% sostituire a XX il totale ore della settimana
\begin{itemize}
    \item \textbf{Prima Settimana (24/06/2024 - 28/06/2024, 34 ore)}
    \begin{itemize}
        \item Familiarizzazione con l'azienda e il team;
        \item Partecipazione a presentazioni introduttive sui concetti di gestione aziendale, database, machine learning, grafi e process mining;
        \item Studio del materiale fornito sul processo aziendale e sugli strumenti ERP utilizzati;
        \item Inizio della documentazione sulle piattaforme di Process Mining e Data Mining.
        \item Retrospettiva settimanale.
    \end{itemize}

    \item \textbf{Seconda Settimana (01/07/2024 - 05/07/2024, 34 ore)}
    \begin{itemize}
        \item Approfondimento sui concetti di machine learning applicati ai dati di processo;
        \item Studio dei grafi e delle loro applicazioni nel process mining;
        \item Ricerca e lettura di articoli e documenti sulle piattaforme di Process Mining (es. Celonis);
        \item Preparazione di una relazione di sintesi sulle nozioni apprese, con particolare attenzione agli strumenti e alle tecniche di process mining.
        \item Retrospettiva settimanale.
    \end{itemize}

    \item \textbf{Terza Settimana (08/07/2024 - 12/07/2024, 34 ore)}
    \begin{itemize}
        \item Installazione e configurazione dell'ambiente di lavoro con Celonis;
        \item Studio approfondito delle funzionalità di Celonis, con particolare attenzione all'interfacciamento con i database ERP;
        \item Inizio della documentazione su come estrarre e caricare i dati in Celonis.
        \item Retrospettiva settimanale.
    \end{itemize}

    \item \textbf{Quarta Settimana (15/07/2024 - 19/07/2024, 34 ore)}
    \begin{itemize}
        \item Esercitazioni pratiche sull'estrazione di dati da un database ERP simulato e il caricamento degli stessi in Celonis;
        \item Analisi dei dati caricati e comprensione delle principali metriche e KPI forniti dalla piattaforma;
        \item Creazione di semplici dashboard in Celonis per visualizzare i flussi di processo.
        \item Retrospettiva settimanale.
    \end{itemize}

    \item \textbf{Quinta Settimana (22/07/2024 - 26/07/2024, 34 ore)}
    \begin{itemize}
        \item Approfondimento su tecniche avanzate di analisi dei dati con Celonis;
        \item Creazione di report dettagliati sui processi aziendali, inclusi flussi di operazioni, tempi di attraversamento e identificazione di colli di bottiglia;
        \item Discussione con il tutor dei risultati ottenuti e delle eventuali problematiche riscontrate.
        \item Retrospettiva settimanale.
    \end{itemize}
\newpage

    \item \textbf{Sesta Settimana (29/07/2024 - 02/08/2024, 34 ore)}
    \begin{itemize}
        \item Caricamento di dati reali provenienti dall'ERP aziendale in Celonis;
        \item Verifica e pulizia dei dati caricati per garantire la loro qualità e coerenza;
        \item Inizio della creazione di modelli di processo basati sui dati reali.
        \item Retrospettiva settimanale.
    \end{itemize}


    \item \textbf{Settima Settimana (05/08/2024 - 09/08/2024, 34 ore)}
    \begin{itemize}
        \item Affinamento dei modelli di processo creati, con correzione di eventuali criticità riscontrate;
        \item Inizio dell'analisi dettagliata dei risultati ottenuti dai modelli, con identificazione di inefficienze e punti di miglioramento;
        \item Discussione con il tutor dei risultati intermedi e delle possibili correzioni da apportare.
        \item Retrospettiva settimanale.
    \end{itemize}

    \item \textbf{Ottava Settimana (19/08/2024 - 23/08/2024, 34 ore)}
    \begin{itemize}
        \item Ulteriore affinamento dei modelli e correzione delle criticità;
        \item Creazione di report avanzati e dashboard interattive per la visualizzazione dei risultati;
        \item Discussione dei risultati con il team, con particolare attenzione alle conclusioni preliminari e ai possibili sviluppi futuri.
        \item Retrospettiva settimanale.
    \end{itemize}

    \item \textbf{Nona Settimana (26/08/2024 - 30/08/2024, 34 ore)}
    \begin{itemize}
        \item Preparazione della relazione finale con una sintesi dettagliata dei risultati ottenuti;
        \item Presentazione delle caratteristiche, pregi e difetti delle piattaforme utilizzate;
        \item Conclusioni finali e suggerimenti per miglioramenti futuri;
        \item Discussione con il tutor e il team delle competenze acquisite e delle prospettive di crescita futura.
        \item Retrospettiva settimanale.
    \end{itemize}
\end{itemize}
}

% Indicare il totale complessivo (deve essere compreso tra le 300 e le 320 ore)
\newcommand{\totaleOre}{306}

\newcommand{\obiettiviObbligatori}{
	 \item \underline{\textit{O01}}: Acquisire conoscenze di base su sistemi gestionali, database, machine learning, grafi e process mining;
	 \item \underline{\textit{O02}}: Approfondimento documentale delle piattaforme di Process Mining e Data Mining, con particolare attenzione a Celonis;
	 \item \underline{\textit{O03}}: Installazione, configurazione e comprensione dell'interfacciamento con la piattaforma Celonis;
	 \item \underline{\textit{O04}}: Estrarre e caricare dati dall'ERP aziendale o generati ad-hoc nelle piattaforme di Process Mining;
	 \item \underline{\textit{O05}}: Pulizia e trasformazione dei dati per adattarli ai requisiti delle piattaforme;
	 \item \underline{\textit{O06}}: Creare e affinare modelli di processo basati sui dati reali caricati nelle piattaforme;
	 \item \underline{\textit{O07}}: Analisi dei dati processati dalle piattaforme e interpretazione dei risultati con l'aiuto del tutor;
	 \item \underline{\textit{O08}}: Redigere una relazione finale con spiegazioni, osservazioni, pregi e difetti delle piattaforme utilizzate;
	 \item \underline{\textit{O09}}: Documentare i risultati ottenuti e proporre possibili sviluppi futuri basati sulle analisi effettuate;
}

\newcommand{\obiettiviDesiderabili}{
	 \item \underline{\textit{D01}}: Identificare punti di inefficienza nei processi aziendali e suggerire possibili miglioramenti;
	 \item \underline{\textit{D02}}: Misurare le performance delle piattaforme in termini di velocità di elaborazione, precisione dei risultati e usabilità;
}

\newcommand{\obiettiviFacoltativi}{
	 \item \underline{\textit{F01}}: Studio e utilizzo di piattaforme di Process Mining diverse da Celonis;
	 \item \underline{\textit{F02}}: Approfondire conoscenze avanzate di data analysis e machine learning;
}
