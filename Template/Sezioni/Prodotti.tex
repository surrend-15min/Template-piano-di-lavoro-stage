%----------------------------------------------------------------------------------------
% DESCRIPTION OF THE PRODUCTS THAT ARE BEING EXPECTED FROM THE STAGE
%----------------------------------------------------------------------------------------
\section*{Prodotti attesi}
% Personalizzare definendo i prodotti attesi (facoltativo)
Lo studente dovrà produrre una relazione scritta che illustri i seguenti punti.

\begin{enumerate}
\item Diario dell'esperienza \
Un diario dettagliato dell'esperienza svolta durante lo stage, documentando i progressi (e i regressi) effettuati settimanalmente, gli obiettivi parziali raggiunti e le sfide incontrate. Questo diario servirà a tenere traccia delle attività svolte e a riflettere sul processo di apprendimento.

\item Relazione sulle piattaforme di Process Mining \
Una relazione completa che descriva le caratteristiche e l'impiego delle piattaforme di Process Mining studiate. La relazione dovrà includere:
\begin{itemize}
\item Una panoramica delle funzionalità delle piattaforme.
\item I vantaggi e gli svantaggi riscontrati durante l'uso.
\item Considerazioni sull'interfacciamento con i database ERP.
\end{itemize}

\item Prototipi funzionanti \
Prototipi funzionanti realizzati utilizzando le piattaforme di Process Mining, con dati estratti e caricati dall'ERP aziendale. I prototipi dovranno includere:
\begin{itemize}
\item Modelli di processo basati sui dati reali.
\item Dashboard interattive per la visualizzazione dei flussi di processo, dei tempi di attraversamento e dei colli di bottiglia.
\item Report dettagliati sui processi aziendali analizzati.
\end{itemize}

\item Analisi dei risultati \
Un documento che presenta un'analisi approfondita dei risultati ottenuti dai prototipi funzionanti, includendo:
\begin{itemize}
\item Interpretazione delle metriche e dei KPI forniti dalle piattaforme.
\item Identificazione di inefficienze e suggerimenti per miglioramenti.
\item Conclusioni tratte dall'analisi e potenziali sviluppi futuri.
\end{itemize}

\item Documentazione dei casi d'uso \
Una documentazione dettagliata dei casi d'uso concretizzati, che descriva il processo di estrazione dei dati, il caricamento nelle piattaforme di Process Mining e i report prodotti. Questa documentazione dovrà includere:
\begin{itemize}
\item Descrizione dei dati utilizzati e dei processi analizzati.
\item Risultati ottenuti dall'analisi dei dati.
\item Conclusioni sui possibili miglioramenti ai processi aziendali sottoposti ad analisi.
\end{itemize}

\end{enumerate}

Nel qual caso in cui lo studente, in seguito all'analisi, abbia ancora tempo a sua disposizione, potrà approfondire ulteriormente l'analisi dei processi aziendali, esplorando ulteriori funzionalità delle piattaforme di Process Mining e contribuendo alla definizione di strategie di miglioramento per i processi analizzati.
