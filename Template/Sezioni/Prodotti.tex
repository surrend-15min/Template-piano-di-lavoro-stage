%----------------------------------------------------------------------------------------
%	DESCRIPTION OF THE PRODUCTS THAT ARE BEING EXPECTED FROM THE STAGE
%----------------------------------------------------------------------------------------
\section*{Prodotti attesi}
% Personalizzare definendo i prodotti attesi (facoltativo)
Lo studente dovrà produrre una relazione scritta che illustri i seguenti punti.
\begin{enumerate}
	\item Studio e interfacciamento con Celonis \\
Descrizione dettagliata dell'interfacciamento con la piattaforma Celonis, includendo le funzionalità principali studiate e le prime esperienze pratiche di utilizzo. Vanno incluse osservazioni su eventuali sfide incontrate e soluzioni adottate.

	\item Caricamento e affinamento dei dati \\
Relazione sul processo di caricamento dei dati nell'applicativo, descrivendo le tecniche e gli strumenti utilizzati per il raffinamento dei dati per adattarli alle esigenze della piattaforma. Questa sezione deve anche includere eventuali problemi di qualità dei dati riscontrati e le strategie adottate per risolverli.

	\item Analisi dei risultati e conclusioni \\
Analisi dei risultati ottenuti dalle piattaforme di Process Mining utilizzate. Vanno presentate le rappresentazioni grafiche dei processi, i dati aggregati e analitici, e una valutazione critica delle performance delle piattaforme. Includere anche possibili sviluppi futuri e raccomandazioni per migliorare l'uso degli strumenti di Process Mining.

	\item Prototipi di valutazione \\
Descrizione dei prototipi realizzati per valutare le piattaforme di Process Mining, basata sui dati estratti o generati. Spiegare i criteri di valutazione adottati, i risultati dei test e le conclusioni tratte sull'efficacia delle piattaforme.

	\item Approfondimento documentale \\
Resoconto delle attività di studio delle piattaforme ipotizzate, con un'analisi comparativa delle loro funzionalità, vantaggi e svantaggi. Includere anche le risorse documentali utilizzate e le principali lezioni apprese durante questa fase.

	\item Raccomandazioni per l'implementazione futura \\
In base ai risultati ottenuti, suggerimenti pratici per l'implementazione futura di piattaforme di Process Mining in azienda, identificando le migliori pratiche e le aree di miglioramento.

\end{enumerate}

La relazione scritta dovrà essere corredata delle opportune spiegazioni e osservazioni, mettendo in evidenza le caratteristiche e i pregi (o i difetti) delle piattaforme impiegate.

Nel qual caso in cui lo studente, in seguito all'analisi, abbia ancora tempo a sua disposizione, potrà approfondire ulteriori aspetti delle piattaforme studiate o esplorare nuove tecnologie emergenti nel campo del Process Mining e del Data Mining.