%----------------------------------------------------------------------------------------
%	STAGE DESCRIPTION
%----------------------------------------------------------------------------------------
\section*{Scopo dello stage}
% Personalizzare inserendo lo scopo dello stage, cioè una breve descrizione
Lo stage si svolge nell’ambito della verifica di fattibilità di impiego di alcune piattaforme commerciali specifiche di Process Mining, o più genericamente di Data Mining, al fine di dedurre le caratteristiche dei processi aziendali (in termini di flussi di operazioni, tempi di attraversamento, colli di bottiglia, ecc) a partire dai dati di tracciatura delle attività svolte dagli utenti mediante l’impiego delle applicazioni dell’ERP, oppure delle attività automatizzate svolte dai servizi di quest’ultimo.
L’intento è di ottenere dai dati di processo una rappresentazione, anche nelle modalità grafiche che tali piattaforme sono in grado di rappresentare (a grafo o a simili ai diagrammi BPM), dei processi che si svolgono in azienda impiegando l’ERP, e una raccolta di dati aggregati e analitici circa il funzionamento di tali processi.
Queste analisi sono utili a comprendere non solo i reali processi che hanno luogo rispetto a quelli immaginati oppure stabiliti sulla “carta”, ma anche i punti di inefficienza e le rispettive cause, e quindi di miglioramento dei processi stessi.

Lo stagista sarà affiancato da un tutor che ha competenze di data analysis, che aiuterà a comprendere la quantità e la qualità dei dati necessari per alimentare i prototipi, a interpretare i risultati ottenuti dagli strumenti impiegati, nonché fornire le nozioni di base riguardanti il data mining applicato ai casi dei dati di processo.

